\documentclass{article}
\usepackage{graphicx} % Required for inserting images

\title{18.02 HW2}

\begin{document}

\maketitle

\begin{enumerate}
    \item Problem 2(b)
    
        Assume $\mathcal{P}$ is countable. Then there exists a bijection $f: \mathbb{N} \to \mathcal{P}$. \\ 
        Let $D = \{n \in \mathbb{N} : n \notin f(n)\}$. Note that $D$ is also a subset of $\mathbb{N}$.
        \\
        Hence $\exists N \in \mathbb{N}$ s.t. $f(N) = D$.
        \\
        We will now prove that there is contradiction no matter we assume $N \in D$ or $N \notin D$.
        \\
        \\
        Suppose $N \in D$. Then $N \in f(N)$. Hence $N \notin D$ and causing contradiction.
        \\
        Suppose $N \notin D$. Then $N \notin f(N)$ and hence $N \in D$, causing contradiction.
        \\
        \\
        Hence $\mathcal{P}$ is not countable.


    \item Problem 3
    
        (a) First we prove that $\bar{B_n} \subseteq \bigcup_{i=1}^{n} \bar{A_i}$.
        \\
        \\
        Let $x \in \bar{B_n}$. If $x \in B_n$ it is trivial. Assume $x$ is a limit point of $B_n$. Then $\forall m \in \mathbb{N} N_{\frac{1}{m}}(x)$ contains a point $x_0$ not equal to $x$ and $x_0 \in \bar{B_n}$. Since $A_i$ is finite collection, $\exists k$ s.t. for infinite number of $m$ $N_{\frac{1}{m}}(x)$ contains a point not equal to $x$ and belongs to $A_k$. So $x \in \bar{A_k}$ and hence $x \in \bigup_{i=1}^{n}\bar{A_i}$, induces that $\bar{B_n} \subseteq \bigcup_{i=1}^{n} \bar{A_i}$.
        \\
        \\
        $\bigcup_{i=1}^{n} \bar{A_i} \subseteq \bar{B_n}$ is trivial.
        \\
        \\
        (b) Consider the sequence of set $(1/2, 1), (1/4, 1/2), (1/8, 1/4), ...$ as $A_i$.
\end{enumerate}

\end{document}

